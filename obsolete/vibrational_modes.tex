Like everything in the universe the mechanical resonator is also vibrating due to its finite temperature. To understand the mechanical behaviour of the resonator we need to determine its vibrational modes. Luckily we are only considering a nanomechanical resonator, also commonly called membrane which dimensions can be considered as a 2D plane, because its length $l$ is much larger than its thickness $d$, i.e.  $l \gg d$. A 3D model is also possible, but would require that we look into different strains and stresses in different depths of the material, complicating the maths unnecessary. My experience is that the 2D model is a good approximation for identifying different mechanical modes in a spectrum \textbf{(cite Bachelors project)}.

We start out by writing the 2D wave equation,
\begin{align}
\frac{\partial^2 z }{\partial t^2} = C^2\nabla^2 z
\end{align}
where $z$ is the displacement from equilibrium out of the plane and $C$ is the propagation speed of the wave.