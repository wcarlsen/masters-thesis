Traditionally, we associate experiments in quantum mechanics with some of the smallest constituents of matter, such as atoms or ions. In resent years macroscopic objects have proven to be a promising testbed of quantum mechanics, due to significant advances within material science and fabrication. As a result, the field of optomechanics has sprung and gained considerable momentum in resent years. Among experiments in fundamental quantum optomechanics are ground state cooling of macroscopic mechanical object \cite{oconnell2010, chan2011, teufel2011}, squeezing of light in an optomechanical system \cite{purdy2013, safavi2013}, back-action evation \cite{suh2014} and many others.

Optomechanical systems come in many flavours and span from optomechanical crystals \cite{chan2011}, whispering-gallery-mode microresonators \cite{schliesser2009}, to superconducting circuits \cite{teufel2011, oconnell2010, suh2014} and nanomechanical resonators inside standard Fabry-Perot optical cavities \cite{jayich2008}. The system presented in this work belongs to the latter category and is commonly referred to as the {\it{membrane-in-the-middle}} system. The system was pioneered by the group of Jack Harris in 2008 \cite{jayich2008}. It combines one of the most studied type of optical cavities with one of the most promising mechanical resonators, namely highly stressed silicon nitride membranes, ensuring high optical, as well as mechanical, quality factors. Despite great progress towards the ground state, this goal has yet to be achieved for this particular system. The work present in this thesis describes the membrane-in-the-middle system at the Niels Bohr Institute and our quest for ground state cooling.