This section is a description of the experiment on how well our optomechanical membrane-in-the-middle system, at the current state, can cool the nanomechanical membrane's (3,3)-mode starting at cryogenic temperatures. The (3,3)-mode has shown high quality factors at room temperature and has promising optomechanical single-photon coupling strength. The complete experiment consist of four smaller sub-experiments. We will in detail describe these sub-experiments and show their results. We will start out by measuring the mechanical ringdown time by optical excitation, to confirm high quality factors at cryogenic temperatures, then we will characterize the membrane-in-the-middle system by making a cavity resonance modulation map and by measuring the linewidth of the cavity. After these initial characterization measurements we will do a (red) detuning series, where we try to lock the laser to the cavity at high input power and close to cavity resonance as possible. We record an optomechanical-induced-transparency spectrum and a regular spectrum while we slowly red detune away from cavity resonance, to obtain the mean phonon occupancy for the mechanical mode.

Before we go into the description of how the spectra were obtained, we make a hand waving argument of why we are able to detect the membrane fluctuations. First let us consider a bare cavity transform phase fluctuations into amplitude fluctuations. If we place a membrane inside such a cavity, then the membrane fluctuations cause phase modulation of the light, which then will be transformed into an amplitude fluctuation. The amplitude fluctuations of the light are picked up by a detector. The detector converts the power fluctuations into a small fluctuating current via the photoelectric effect, which is then converted into a fluctuating voltage. The voltage is what we at the end of the day see as our signal on the oscilloscope. All of these conversions are considered to be linear for small fluctuations. We can also from a simple cavity response consideration predict when an optomechanical cavity is most sensitive to small phase fluctuations, which is when we are detuned by half of the cavity linewidth, because the slope of the cavity response (derivative of Lorentzian) is the steepest at exactly this detuning.