Most optomechanical systems try to minimize losses, both mechanical and optical. Another parameter that is of particular interest is mass or rather effective mass of the mechanical oscillator, because smaller masses enhances the interaction between optical and mechanical degree of freedom. It can be thought of as a single photon having a higher impact, because of the smaller effective mass. But small mass and great optical properties, such as high reflectivity, do not always commute. ``Nano/micro" membranes do not possess a high enough reflectivity, such that one can reach the high finesse limit with only a high reflective mirror and a membrane. Normally membranes do not have reflectivities $\left| r_m \right|^2$ exceeding 0.5, even though attempts are being made by stacking them in layers. This kind of ruin the plan of creating an optomechanical system with low optical loss. But enclosing an ultra-thin dielectric film between two mirrors of an already high finesse optical cavity solves the problem. We call it the membrane-in-the-middle (or MIM) system \cite{thompson2008,jayich2008} and is depicted in figure \ref{fig:transfer_model}.
The membrane-in-the-middle system is one way to achieve a trade-off between great mechanical performance and low optical loss. Mirrors which enable high finesse are commercially available at finesses $\mathcal{F} > 10^6$ and mechanical resonators are also commercially available \cite{jayich2008} or can be fabricated ``in-house" with very high quality factors $Q > 10^6$ \cite{tsaturyan2014}.