The quality factor, which is often refered to as $Q$, is a measure of how well energy is stored in the oscillator before being lost due to dissipation mechanisms. Our membranes is a good example of an oscillator with high quality factors (the same holds for our high finesse cavity). We often see $Q > 10^6$ at cryogenic temperatures even if the frame is clamped. Mathematically we define the quality factor as

\begin{equation}
Q \equiv \frac{\Omega_m}{\Gamma_m},
\end{equation}
\noindent
where $\Gamma_m$ in Fourier domain is the full-width-half-maximum of $S_{xx}$ as shown in figure \ref{fig:mem_temp}. In the time-domain we already showed that the mechanical damping is inversely proportional to the ringdown-time $\tau$, which is the exponential decay of the oscillation amplitude. For very high $Q$'s it is customary to use ringdowns as a primary tool to obtain $\Gamma_m$, since $Q$'s on the order of one million at mechanical resonances of \SI{}{\mega\hertz}, yields \SI{1}{\hertz} FWHM. In this regime you are likely to be limited by the resolution bandwidth of your spectrum analyzer.

Since mechanical quality factors are of great importance in this field of study, many study the different dissipation mechanisms in membranes. It is not the scope of this work and we will refer to \cite{barg2014} for a more detailed discussion of membrane dissipation mechanisms.