During the past years we have set up an optomechanical membrane-in-the-middle system towards showing a quantum enabled system, by squeezing of light or cooling a nanomechanical resonator to its motional ground state. The work presented here describes our efforts towards ground state cooling.

The setup consist of a high finesse optical Fabry-Perot cavity with a highly stressed silicon nitride nanomechanical resonator placed inside. The nanomechanical resonator is shielded from the environment by a phononic bandgap structure, but due to complications of low frequency modes of the collective structure, the sample is returned to the state of a regular nanomechanical resonator, i.e. by damping the bandgap structure within which the membrane is suspended. A sample holder is designed to achieve good thermalization of the nanomechanical resonator. The system is characterized at cryogenic temperatures in pursuit of optical sideband cooling of a mechanical mode to the ground state.

We have achieved a finesse of $\sim56\mathrm{k}$ (corrosponding to an optical linewidth of $\sim$\SI{1.6}{\mega\hertz}) with a high-Q ($\sim6\times10^6$) membrane embedded inside the optical cavity. The single photon coupling for the (3,3)-mode of the membrane is $\sim$\SI{88}{\hertz} (measure by means of optomechanical induced transparency) and the effective mechanical broadening achieved with the highest optical powers was measured to be $\sim$\SI{5.5}{\kilo\hertz}. As such, we were able to record a minimum phonon occupancy of $27\substack{+11 \\ -7}$, corresponding to an effective temperature of $\sim$\SI{3}{\milli\kelvin}.