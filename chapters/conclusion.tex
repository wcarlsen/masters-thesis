We set out to rigorously test the cooling performance of our optomechanical membrane-in-the-middle system, showing with some precision that we can cool a mechanical mode down to a mean phonon occupancy of $\bar{n}_{phonon}^{min} \approx 27\substack{+11 \\ -7}$, corrosponding to an effective temperature of $T_{eff} \approx$\SI{3}{\milli\kelvin}. While not being in the ground state we can still appreciate the low phonon occupancy and the extremely low temperature. We can for example calculate the probability of being in the ground state $P_{n = 0} \approx 0.04$ given our current occupancy \cite{gerry2005}. We are actually in the ground state 4\% of the time. From equation \eqref{eq:nf} using our phonon occupancy we can extract the bath temperature we couple to, or in other words how well we thermalize the membrane through the sample holder. The bath temperature is found to be unexpectedly high, yielding $T_{bath} \approx$ \SI{43}{\kelvin}. This is unexpected since the cold finger is measured to be \SI{4}{\kelvin} (using the built-in temperature sensor in the cryostat) and at least in theory our sample holder should perform reasonably well. We have also from initial tests of thermalization seen indications of a membrane bath temperature of roughly \SI{10}{\kelvin} to \SI{30}{\kelvin}. Nonetheless, we can only draw two conclusions from the extracted bath temperature, either our sample holder do not thermalize the membrane or our spectrum is calibrated wrong and therefore estimates the phonon occupancy to be to high. Looking at the mentioned not included errors from the calibration process at the end of section \ref{sec:ex_spec}, we wrote that we neglected the corrections from the transduction function of the signal sent to the EOM. What lies behind the reasoning for excluding this is that the electronic line from the source to the EOM is unknown. We think that we had two power splitters, but we do not know which. Therefore, we assumed the osses to be \SI{6}{dB} and this could very well be a wrong assumption. The given EOM power in this work is the least generous for the phonon occupancy. If the power sent to the EOM was lower it would in fact correspond to a lower phonon number. The other neglected uncertainty is the mean intra-cavity photon number calculated from the detector's DC level. The model in \eqref{eq:cav_circ_pwr} is too simple to use and should have included a correction factor according from the transfer matrix model, because of the two sub-cavities not containing the same number of photons depending on the membrane position. We do not know in which direction it will shift the phonon occupancy and these considerations are therefore not included in this work.

As a final remark we will try to estimate weather or not the ground state is within reach for our setup. A parameter we know can be improved is the mechanical quality factor, as quality factors above ten million have been reached for some of our samples. This improvement would decrease the final phonon occupancy roughly a factor of two. If we trust the measured minimum phonon occupancy, we know that the main hindrance of reaching the ground state is the thermalization issue\footnote{A new improved generation of the sample holder has been machined and is now already installed in the lab.}. If thermalization can be improved by a factor of five, the future would look very bright. It is of course also possible to look for other mechanical modes with different couplings and thermal occupations. It thus seems that the ground state is indeed within reach.